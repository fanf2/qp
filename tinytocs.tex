\documentclass{tinytocs}

\usepackage{url}

\title{QP tries are smaller and faster\\
  than crit-bit trees}

\author{Tony Finch\\
  \affaddr{University of Cambridge}\\
  \email{dot@dotat.at}}

\begin{document}

\maketitle

\abstract{

  A trie data structure stores an ordered set of keys; the branching
  structure of a trie depends on the lexical properties of its keys
  independent of the order of insertion. Compact implementations of
  PATRICIA binary tries called crit-bit trees \cite{djb} have just two
  words of overhead per item stored.

  A hash array mapped trie (HAMT) \cite{bagwell} has wide fan-out,
  indexing each tree node using several key bits; each node is
  compressed using the population count of a bitmap to omit NULL child
  pointers. Bagwell sketches an un-hashed pure trie variant of HAMT in
  section 5 but doesn't eliminate redundant single-child nodes like
  crit-bit trees.

  Our contribution, QP tries \cite{qp}, are similar to crit-bit trees
  but test 5 bits per indirection instead of 1, using the HAMT bitmap
  POPCNT trick to keep overhead to at most two 64 bit words per item.
  QP tries prefetch the child pointer array while calculating which
  child is next; this reduces indirection latency and increases
  performance by about 5\%. QP tries have variable-sized nodes, so
  stress memory allocation more than crit-bit tries, but are much
  cheaper in other respects.

  We created similar implementations of QP tries and crit-bit trees,
  and benchmarked them using lists of: English words; identifiers in
  the BIND9 source code; domain names from a university; Alexa top
  million domain names. We measured average: trie depth; space
  overhead per item; mutation and search time.

}

\tinybody{Typical QP trie\\
depth is 0.35-0.40\\
space is 0.5-0.6\\
time is 0.6-0.8\\
of equivalent crit-bit tree.}

\bibliographystyle{abbrv}
\bibliography{tinytocs}

\end{document}
