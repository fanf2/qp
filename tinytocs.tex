\documentclass{tinytocs}

\usepackage{url}

\title{QP tries are smaller and faster\\
  than crit-bit trees}

%\author{Tony Finch\\
%  \affaddr{University of Cambridge}\\
%  \email{dot@dotat.at}}

\begin{document}

\maketitle

\abstract{

  Compact implementations of PATRICIA tries called crit-bit trees
  \cite{djb} have two words of overhead per item stored. HAMTs
  \cite{bagwell} index tree nodes using multiple key bits; they use a
  bitmap and POPCNT to omit NULL child pointers from tree nodes.
  Bagwell sketches a trie variant of HAMTs in section 5 but doesn't
  skip redundant nodes like crit-bit trees.

  QP tries \cite{qp} are similar to crit-bit trees but test 5 bits per
  indirection instead of 1, using the HAMT bitmap POPCNT trick to keep
  overhead to at most two 64 bit words per item. QP tries prefetch the
  child node array while calculating which child is next; this reduces
  indirection latency and increases performance by about 5\%.

  We created similar implementations of QP tries and crit-bit trees,
  and benchmarked them using lists of: English words; identifiers in
  the BIND9 source code; domain names from a university; Alexa top
  million domain names. We measured average: trie depth; space
  overhead per item; mutation and search time.

}

\tinybody{Typical QP trie\\
depth is 0.35-0.40\\
space is 0.5-0.6\\
time is 0.6-0.8\\
of equivalent crit-bit tree.}

\bibliographystyle{abbrv}
\bibliography{tinytocs}

\end{document}
